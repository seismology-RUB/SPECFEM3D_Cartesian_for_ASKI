%-----------------------------------------------------------------------------
%   Copyright 2013 Florian Schumacher
%
%   This file is part of the SPECFEM3D_Cartesian_for_ASKI manual as a LaTeX 
%   document with main file SPECFEM3D_Cartesian_for_ASKI_manual.tex
%
%   Permission is granted to copy, distribute and/or modify this document
%   under the terms of the GNU Free Documentation License, Version 1.3
%   or any later version published by the Free Software Foundation;
%   with no Invariant Sections, no Front-Cover Texts, and no Back-Cover Texts.
%   A copy of the license is included in the section entitled ``GNU
%   Free Documentation License''. 
%-----------------------------------------------------------------------------
%
\documentclass[12pt,a4paper]{article}

\usepackage[english]{babel} %language selection
\selectlanguage{english}

\pagenumbering{arabic}

\usepackage[affil-it]{authblk}
\usepackage{times} % 'times new roman' script style

%\usepackage{amsmath}
%\usepackage{amssymb}
%\usepackage{graphicx}

% use package url with [obeyspaces] in order to correctly display \nolinkurl WITH spaces 
%(used in \newcommand{\lcode} below). As hyperref internally loads package url, you can pass
% option obeyspaces of package url to package hyperref as follows
\PassOptionsToPackage{obeyspaces}{url}\usepackage{hyperref}
%\hypersetup{colorlinks, 
%           citecolor=black,
%           filecolor=black,
%           linkcolor=black,
%           urlcolor=black,
%           bookmarksopen=true,
%           pdftex}
%\hfuzz = .6pt % avoid black boxes

% the following is an ugly solution of allowing line breaks in urls additionally after every normal 
% alphabetic character which (if \nolinkurl is used in \newcommand{\lcode} below) at all allows line 
% breaks of long routine names like 'transformToStandardCellInversionGrid', BUT of course also breaks
% any other term formatted by \lcode at any character, which is maybe not very nice.
\let\origUrlBreaks\UrlBreaks
%\renewcommand*{\UrlBreaks}{\origUrlBreaks\do\a\do\b\do\c\do\d\do\e\do\f\do\g\do\h\do\i\do\j\do\k\do\l\do\m\do\n\do\o\do\p\do\q\do\r\do\s\do\t\do\u\do\v\do\w\do\x\do\y\do\z\do\A\do\B\do\C\do\D\do\E\do\F\do\G\do\H\do\I\do\J\do\K\do\L\do\M\do\N\do\O\do\P\do\Q\do\R\do\S\do\T\do\U\do\V\do\W\do\X\do\Y\do\Z}


%% POSSIBLE PACKAGES TO DISPLAY CODE
%%
%% package alltt: verbatim environment within which math is displayed correctly
%% usage: \begin{alltt}\end{alltt}
%\usepackage{alltt}
%%
%% package listings: provides environments to display code fragments (with a lot of special characters) in a more evolved fashion than verbatim (alltt)
%% only uncomment (both next lines), if used in \newcommand{\lcode} below
%\usepackage{listings}
%\lstset{basicstyle =\ttfamily}%\small}

\usepackage[paperwidth=21.0cm,paperheight=29.7cm, left=2.5cm,right=2.5cm,top=2.0cm,
            bottom=2.0cm,headheight=0in,footskip=1.0cm]{geometry}
%-------------------------------
%
% COMMANDS FOR IN-LINE PHRASES IN CODE-STYLE
%
%%% ttfamily does not properly support any special characters
%\newcommand{\lcode}[1]{ {\ttfamily #1 }}
%
%%% lstinline is a good solution, in general, but it makes problems in line breaks!
%\newcommand{\lcode}[1]{\lstinline[breaklines=true]$#1$}
%
%%% although there are no actual links, it uses the same font as lstinline (when \lstset{basicstyle =\ttfamily}), 
%%% but produces better line breaks!
\newcommand{\lcode}[1]{\nolinkurl{#1}}
%
%%% need \lcodetitle, since \nolinkurl in a title of a numerated (sub)section (not *) causes problems in bookmark 
%%% view in adobe reader (why?! what is the actual problem?), \lcodetitle, however, does NOT support stuff like '_' etc.
\newcommand{\lcodetitle}[1]{ {\ttfamily #1} }
%
%
\newcommand{\ASKI}{ {\ttfamily ASKI} }
%
%
% OTHER NEW COMMANDS
%
\newcommand{\inotice}[1]{ \fbox{\parbox[t]{0.9\textwidth}{{\bf Important:} \\#1}} }
\newcommand{\notice}[1]{ \fbox{\parbox[t]{0.9\textwidth}{#1}} }
\newcommand{\myref}[1]{\ref{#1} (page~\pageref{#1})}
\newcommand{\myaref}[1]{$\rightarrow$~\ref{#1} (page~\pageref{#1})}
%
%-------------------------------
%
% END OF PREAMBLE
%####################################################################
%
\begin{document}
%
\setlength{\parindent}{0cm}
\addtolength{\parskip}{0.1cm}
% TeX’s first attempt at breaking lines is performed without even trying hyphenation: 
% TeX sets its “tolerance” of line breaking oddities to the internal value \pretolerance
% an “infinite” tolerance is represented by the value 10000, but may lead to very bad line breaks indeed!
%\pretolerance=10000
%
%-------------------------------
% TITLE PAGE
%
% without \usepackage[affil-it]{authblk} e.g.:
%\author{Florian Schumacher \thanks{\texttt{florian.schumacher@rub.de}; corresponding author} \and Wolfgang Friederich \thanks{\texttt{wolfgang.friederich@rub.de}}}
%
\title{Using {\tt \Huge SPECFEM3D\_Cartesian-2.1} for \\ \tt {\Huge ASKI} {\rm--} {\Huge A}{\large nalysis of} {\Huge S}{\large ensitivity \\ and} {\Huge\tt K}{\large ernel} {\Huge\tt I}{\large nversion, version 0.3} }
%\author[1]{Florian Schumacher \thanks{\texttt{florian.schumacher@email.address}; corresponding author}}
\author[1]{Florian Schumacher}
\author[1]{Wolfgang Friederich}
\affil[1]{Ruhr-Universit\"at Bochum} % for this you need \usepackage[affil-it]{authblk}
\date{\today}
%\date{6.12.2004}
%\date{} % no date
\maketitle
%
%-------------------------------
% LICENSE
Copyright \copyright 2013 Florian Schumacher.
Permission is granted to copy, distribute and/or modify this document
under the terms of the GNU Free Documentation License, Version 1.3
or any later version published by the Free Software Foundation;
with no Invariant Sections, no Front-Cover Texts, and no Back-Cover Texts.
A copy of the license is included in the section entitled ``GNU
Free Documentation License''.

\vspace{1cm}

This documentation was written in the hope that it will be useful to the user,
but it \emph{cannot be assured} that it is accurate in every respect or complete in any sense.\\
Please do not hesitate to improve this documentation by incorporating your experiences with 
\lcode{SPECFEM3D for ASKI} and your personal experience of getting used to it. 

Furthermore, my moderate experience with \LaTeX may well give rise to improving the document 
style, hence the readability of the manual as a whole, as well as the coding style of the 
\lcode{.tex} file. 

The \LaTeX source files and all related components of this document are contained in the 
\lcode{SPECFEM3D_Cartesian-2.1 for ASKI 0.3} package, available via \url{http://www.rub.de/ASKI}
\begin{flushright}
Florian Schumacher, 2013
\end{flushright}
%
%-------------------------------
% SECTION Introduction
%#############################################################
\section*{Guide Through This Manual}
%#############################################################
%
We assume that you have sufficient knowledge of how to run the regular \lcode{SPECFEM3D_Cartesian} code.

For details on how to extend the regularly installed \lcode{SPECFEM3D_Cartesian-2.1} code to produce
output for \ASKI, please read section~\ref{extent_to_ASKI}.

Before you start using the code to produce output for \ASKI, please consider the general
comments in section~\ref{general_stuff}.

If you are going to use the automated python script \\ \lcode{run_specfem3dCartesianForASKI_simulations.py} 
then please start reading section~\ref{use_script}. 

If you want to conduct one single simulation producing some output for \ASKI, please start 
reading section~\ref{no_script}.

Section~\ref{file_Par_file_ASKI} is intened to be used as a reference section only.

Bracketed comments starting with ``{\bf TODO IN THE FUTURE:}'' are intended to mark ideas for future work. 
So please ignore if you are just applying the code.
%
%-------------------------------
% TABLE OF CONTENTS
\newpage
\tableofcontents
\newpage
%
%-------------------------------
% SECTION General things
%#############################################################
\section{General Stuff to Consider} \label{general_stuff}
%#############################################################
%
\begin{itemize}
\item parameters \lcode{FILE_KERNEL_REFERENCE_MODEL} and \lcode{FILE_WAVEFIELD_POINTS} 
  of the parameter file for a specific iteration step must be set to some main \ASKI output file,
  which is the basefile name of \lcode{ASKI_outfile} extendet by \lcode{.main}, see \ref{Par_file_ASKI,sub:output}.
  Use the main \ASKI output file of some arbitrary \ASKI output, e.g.\ the kernel displacement output of the first source.
\item As there is a fixed order assumed of \ASKI wavefield points (by procs and local element numbering), the 
  computation of many kernels (e.g.\ for many seismic paths in an inversion) can only be consistent, if the 
  \emph{same} mesh decomposition and the \emph{same} number of procs is used at all times 
  (for those kernels you want to use together, e.g.\ all kernels in your specific iteration step of an inversion).
  It may, hence, be sensible to use the same MPI Databases for all \lcode{SPECFEM3D for ASKI} simulations (adjust your 
  script \lcode{process.sh} in such a way, that you do not always recompile and decompose MESH, but only call the solver
  again (with changed parameter files and source files).
\item there is (probably?!) no proper support for multiple sources in case of producing output for \ASKI.
\item Green tensor simulations are done using  value \lcode{ALT} in the station definition as the 
  \lcode{FORCESOLUTION} \lcode{depth} value, in order to allow receivers to be located not only on the surface.
  It, hence, may be sensible to throughoutly use \lcode{USE_SOURCES_RECVS_Z = .true.} in \lcode{SPECFEM3D/src/shared/constants.h}.
\item As the coordinates of wavefield points (and, hence, inversion grid coordinates) \lcode{SPECFEM3D for ASKI} uses:\\
  First coordinate = \lcode{X}, second coordinate = \lcode{Y}, third coordinate = \lcode{Z}.
\item \dots
\end{itemize}
%
%-------------------------------
% SECTION single simulation without python script
%#############################################################
\section{One Single Simulation} \label{no_script}
%#############################################################
%
As usual, you need to do (external or internal) meshing for the appropriate (current) background model. 
See section \ref{import_model} for details on how to import the current model of an inverision (the model of the last
iteration step) into \lcode{SPECFEM3D}.

Set the regular \lcode{SPECFEM3D} files \lcode{Par_file}, \lcode{CMTSOLUTION / FORCESOLUTION} 
and \lcode{STATIONS} (standard \lcode{SPECFEM3D} functionality; only if you want to record any seismograms).

Additionally, you need to set file \lcode{Par_file_ASKI} to desired values. The file is described 
in detail in section~\ref{file_Par_file_ASKI}. 

After that, you are ready to run the code. As there is no \ASKI dependent change of \lcode{SPECFEM3D} 
components which are sensitive to compiling, you do not need to recompile the \lcode{SPECFEM3D} code every time you
run a \lcode{SPECFEM3D} simulation for \ASKI.
%
%-------------------------------
% SECTION using python script
%#############################################################
\section{Using Automated Python Script for Doing Several Simulations} \label{use_script}
%#############################################################
%
As usual, you need to do (external or internal) meshing for the appropriate (current) background model. 
See section \ref{import_model} for details on how to import the current model of an inverision (the model of the last
iteration step) into \lcode{SPECFEM3D}.

Python script \lcode{run_specfem3dCartesianForASKI_simulations.py} conducts the specified 
kernel simulations (as described inside the script on the top) by iteratively running 
\lcode{SPECFEM3D} simulations, using appropriate settings of the parameter files for each iteration. \\
You need to edit the first lines of the script (above definition of \lcode{class simulation}) and set 
all variables defined there to appropriate values, as described in the comments in the script 
({\bf TODO IN THE FUTURE:} maybe it is better to have an input (file?) mechanism to this script. But then: more
overhead/extra requirements (packages, arguments handling) to cope with on cluster machines \dots)

The python script may not be suitable for the HPC system you are using! If you are not able to adapt 
it in a way which makes it possible to be used, you might have to figure out an analogous way yourself 
how to perform the tasks done by this script.

In case of using the provided python script \lcode{run_specfem3dCartesianForASKI_simulations.py},
some parameters in \lcode{SPECFEM3D} files \lcode{CMTSOLUTION}, \lcode{FORCESOLUTION}, \lcode{Par_file} 
and in file \lcode{Par_file_ASKI} are automatically changed, while the script iteratively conducts 
\lcode{SPECFEM3D} simulations. \\
In the following, only those parameters/lines are listed, which, if necessary, need to be set 
manually before running this python script. All other parameters are set by the script.

%+++++++++++++++++++++++++++++++++++++++++++++++++++++++++++++++++++++++
\subsection{Manually Setting \lcodetitle{Par\_file\_ASKI}}
%+++++++++++++++++++++++++++++++++++++++++++++++++++++++++++++++++++++++
The following \lcode{Par_file_ASKI} parameters need to be set manually before running the python script, 
since they are not changed/set by the script.
\begin{itemize}
\item \lcode{OVERWRITE_ASKI_OUTPUT}
\item \lcode{ASKI_DFT_double}
\item \lcode{ASKI_DFT_apply_taper,ASKI_DFT_taper_percentage}
\item in case of \lcode{define_ASKI_output_volume_by_inversion_grid = False}, you need to manually set
  all parameters concerning the inversion grid, i.e.\ \lcode{ASKI_type_inversion_grid}, 
  \lcode{ASKI_(cw)(xyz)}, \lcode{ASKI_rot_(XYZ)}
\end{itemize}
%
%+++++++++++++++++++++++++++++++++++++++++++++++++++++++++++++++++++++++
\subsection{Manually Setting \lcodetitle{FORCESOLUTION}}
%+++++++++++++++++++++++++++++++++++++++++++++++++++++++++++++++++++++++
%
The python script \emph{always} automatically sets ``latitude:'', ``longitude:'', ``depth:'', 
``factor force source:'', ``component dir vect source E:'', ``component dir vect source N:'', 
``component dir vect source Z\_UP:''. In case of displ and gt simulations, additionally ``hdur:'' is set to ``0.''. 
So, if you wish to do a data simulation for single force sources, using a different ``hdur:'' value, you 
should conduct those in a separate run of the python script.
%
%+++++++++++++++++++++++++++++++++++++++++++++++++++++++++++++++++++++++
\subsection{Manually Setting \lcodetitle{CMTSOLUTION}}
%+++++++++++++++++++++++++++++++++++++++++++++++++++++++++++++++++++++++
%
The python script \emph{always} automatically sets ``latitude:'', ``longitude:'', ``depth:'', 
``Mrr:'', ``Mtt:'', ``Mpp:'', ``Mrt:'', ``Mrp:'', ``Mtp:''. In case of displ and simulations, additionally 
``half duration:'' is set to ``0.''. So, if you wish to do a data simulation for single force sources, 
using a different ``half duration:'' value, you should conduct those in a separate run of the python script.
%
%+++++++++++++++++++++++++++++++++++++++++++++++++++++++++++++++++++++++
\subsection{Manually Setting \lcodetitle{STATIONS}}
%+++++++++++++++++++++++++++++++++++++++++++++++++++++++++++++++++++++++
%
The standard \lcode{SPECFEM3D} \lcode{STATIONS} file should contain the definition of stations as
in the \ASKI file \lcode{file_stations_list}, in consistend \lcode{SPECFEM3D} notation, i.e.\ coordinate
columns being lat ( = Y, third column of \lcode{STATIONS} and fourth column of \lcode{file_stations_list}) and 
lon ( = X, fourth column of \lcode{STATIONS} and third column of \lcode{file_stations_list}) and 
elev ( = Z, sixth column of \lcode{STATIONS} and fifth column of \lcode{file_stations_list}). 

You must also assure to use the very same station names and network codes in file \lcode{STATIONS} as in 
\ASKI file \lcode{file_stations_list}!
%
%-------------------------------
% SECTION importing an external model
%#############################################################
\section{Importing the currently inverted model for next iteration step} \label{import_model}
%#############################################################
%
Exported \lcode{.kim} files (as produced by \ASKI program \lcode{exportKim}) may be read used as a model by 
\lcode{SPECFEM3D for ASKI}, as explained in the following:

The model values defined by the kernel inverted model will be \emph{superimposed} onto the \lcode{SPECFEM3D} 
default model values as defined by e.g.\ \lcode{Cubit}, using a special implementation of the \lcode{SPECFEM3D} 
module \lcode{model_external_values}.

Set \lcode{MODEL = external} in \lcode{Par_file}. Then directory \lcode{DATA} must contain a file
named \lcode{model_external_ASKI} which must contain 3 lines:
\begin{itemize}
\item[line 1] \emph{interpolation type:}\\
  The first line may be of the following formats:
  \begin{itemize}
  \item \lcode{shepard_standard}
    meaning that a standard inverse distance interpolation, respecting for direction of neighbours, after Shepard
    is used to interpolate model values from the inversion grid cell centers onto the new \lcode{SPECFEM3D} GLL points.
    The radii of the inversion grid cells are used to decide whether a GLL point is in range of an inversion grid cell
    or not.
  \item \lcode{shepard_factor_radius  FACTOR}
    whereby \lcode{FACTOR} is a factor to be applied to the inversion grid cell radii before applying the same
    interpolation method as in case \lcode{shepard_standard}.
  \end{itemize}
\item[line 2] \emph{model file type:}\\
  only \lcode{kim_export} supported by now, meaning that the file given in line 3 must be an exported \lcode{.kim} file
  as produced by \ASKI program \lcode{exportKim}.
\item[line 3] \emph{name of model file:}\\
  name of model file of type as defined in line 2, which is expected to be in directory \lcode{DATA}
\end{itemize}
%
%-------------------------------
% SECTION preparing synthetic data
%#############################################################
\section{Preparing Synthetic Data as Expected by \ASKI}
%#############################################################
%
Use binary \lcode{createSpecfem3dSyntheticData}. Calling \lcode{createSpecfem3dSyntheticData -h} will print a
help message how to use it.

It is assumed that a copy of the \lcode{OUTPUT_FILES} folder (without the \lcode{MPI_DATABASES} files etc...)
of all involved \lcode{SPECFEM3D} simulations (which contain the standard seismograms files) can be found at 
the path as choosen by the automated python script (see~\ref{use_script}), i.e.\ filename of the kernel displacement 
file for the respective event with the extension \lcode{_OUTPUT_FILES}. 
The synthetic data then is written in the required form to path \lcode{PATH_SYNTHETIC_DATA}, where the filenames are by
convention \lcode{synthetics_EVENTID_STATIONNAME}.  

The option \lcode{-data} allows you to also transform synthetically created ``real'' data (e.g.\ for a 
pure synthetic inversion) to the required form. Also in this case it is assumed that a copy of the 
\lcode{OUTPUT_FILES} folders can be found at paths as choosen by the automated python script, i.e.\ the base
filename of the measured data for the respective event with extension \lcode{_OUTPUT_FILES}. The ``real'' data 
then is written in the required form to the measured data files with base filenames as defined in the iteration step 
info database. 
%
%-------------------------------
% SECTION preparing measured data
%#############################################################
\section{Preparing synthetically computed ``measured'' data as expected by \lcodetitle{ASKI}}
%#############################################################
%
You can produce \ASKI files for measured data in the required form from \lcode{SPECFEM3D for ASKI} ``data'' 
simulations (e.g.\ produced by automated python script,~\ref{use_script}). This functionality may
be used for synthetic tests, in which you must produce data for some perturbed earth model, which 
is treated as measured data.

Use binary \lcode{createSpecfem3dMeasuredData}. Calling \lcode{createSpecfem3dMeasuredData -h} will print a
help message how to use it.

It is assumed that a copy of the content of the \lcode{OUTPUT_FILES} folder (without the \lcode{MPI_DATABASES} files etc...)
of the ``data'' simulations (which contain the standard seismograms files) can be found at 
directory \lcode{PATH_MEASURED_DATA/data_EVENTID_OUTPUT_FILES}. 
The measured data files then are written in the required form to path \lcode{PATH_MEASURED_DATA}, where the filenames are by
convention \lcode{data_EVENTID_STATIONNAME_COMP}.  
%
%-------------------------------
% SECTION Par_file_ASKI
%#############################################################
\section{File \lcodetitle{Par\_file\_ASKI}} \label{file_Par_file_ASKI}
%#############################################################
%
File \lcode{Par_file_ASKI} is, just like the file \lcode{Par_file}, located in directory 
\lcode{DATA/} of your current \lcode{SPECFEM3D} example. It basically controls \ASKI functionality 
\lcode{SPECFEM3D} if used along with an \ASKI extended \lcode{SPECFEM3D} installation. If in such an 
installation file \lcode{Par_file_ASKI} is not present, no \ASKI output is produced and 
\lcode{SPECFEM3D} runs with standard functionality. 

In the following, we give a short description of the functionality of parameters defined
in file \lcode{Par_file_ASKI}.
%+++++++++++++++++++++++++++++++++++++++++++++++++++++++++++++++++++++++
\subsection{\lcodetitle{ASKI} output} \label{Par_file_ASKI,sub:output}
%+++++++++++++++++++++++++++++++++++++++++++++++++++++++++++++++++++++++
%----------------------------------------------------------------------
\subsubsection*{\lcode{COMPUTE_ASKI_OUTPUT, OVERWRITE_ASKI_OUTPUT}}
%----------------------------------------------------------------------
Parameter \lcode{COMPUTE_ASKI_OUTPUT} controls if any ASKI functionality is applied by \lcode{SPECFEM3D} and
output files (i.e. kernel green tensor kernel displacement files) are produced. If true, \lcode{OVERWRITE_ASKI_OUTPUT}
controls if those files shall be overwritten if existend or not. If false and files exist, the \lcode{SPECFEM3D}
solver will terminate raising an error message.
%----------------------------------------------------------------------
\subsubsection*{\lcode{ASKI_outfile, ASKI_output_ID}}
%----------------------------------------------------------------------
\lcode{ASKI_outfile} defines the absolute base file name of \ASKI output files. \lcode{ASKI_output_ID} is a character
string of maximum lenght as defined by parameter \lcode{length_ASKI_output_ID} in file \lcode{specfem3D_par_ASKI.f90} 
with which all output files of a certain run will be taged, and it will be used to check consistency of the files
(could be a timestamp, eventID, station name, component etc).
%+++++++++++++++++++++++++++++++++++++++++++++++++++++++++++++++++++++++
\subsection{Frequency discretization}
%+++++++++++++++++++++++++++++++++++++++++++++++++++++++++++++++++++++++
The double precision \lcode{df} [Hz] and integer values \lcode{jf} have the following meaning:
The spectra are saved for all frequencies \lcode{f = (jf)*df} [Hz].
%----------------------------------------------------------------------
\subsubsection*{\lcode{ASKI_df, ASKI_nf, ASKI_jf}}
%----------------------------------------------------------------------
\lcode{ASKI_df} is a predefined frequency step that is used to evaluate the spectrum. In case we want to do 
an inverse FT in case of time-domain sensitivity kernel computation, we need to choose \lcode{ASKI_df} with care 
as \lcode{ASKI_df = 1/length_of_time_series} and suitably high frequency indices (dependent on frequency content).
Otherwise we could lose periodicity (if in \lcode{exp^(-i2pi(k)(n)/N)} \lcode{N} is no integer, these are no 
roots of 1 anymore). The spectra are saved for frequencies \lcode{f = (ASKI_jf)*ASKI_df} (\lcode{ASKI_nf} many).
%----------------------------------------------------------------------
\subsubsection*{\lcode{ASKI_DFT_double}}
%----------------------------------------------------------------------
Choose precision of Discrete Fourier Transform. If there is enough memory available, it is highly recommended
to use \lcode{ASKI_DFT_double = .true.} in which case double complex spectra are hold in memory (single precision is 
written to file, though, but less roundoffs during transformation). Otherwise choose \lcode{ASKI_DFT_double = .false.}
in which case single precision spectra will be used in memory. The transformation coefficients \lcode{exp^(-i*2pi*f*t)} 
are always in double complex precision!
%----------------------------------------------------------------------
\subsubsection*{\lcode{ASKI_DFT_apply_taper, ASKI_DFT_taper_percentage}}
%----------------------------------------------------------------------
Decide whether the (oversampled, noisy, ...) time series should be tapered by a hanning taper (on tail)
before (i.e.\ while) applying the discrete fourier transform (on-the-fly). If \lcode{ASKI_DFT_apply_taper = .true.},
the value of \lcode{ASKI_DFT_taper_percentage} (between 0.0 and 1.0) defines the amount of
total time for which the hanning taper will be applied at the tail of the time series.
%+++++++++++++++++++++++++++++++++++++++++++++++++++++++++++++++++++++++
\subsection{Inversion grid}
%+++++++++++++++++++++++++++++++++++++++++++++++++++++++++++++++++++++++
%----------------------------------------------------------------------
\subsubsection*{\lcode{ASKI_type_inversion_grid}}
%----------------------------------------------------------------------
ASKI supports several types of inversion grids for \lcode{FORWARD_METHOD = SPECFEM3D}.
\lcode{ASKI_type_inversion_grid = }
\begin{enumerate}
\item (\lcode{TYPE_INVERSION_GRID = ccsInversionGrid}) \\ 
  NOT TO BE USED WITH SPECFEM3D Cartesian! {\bf NOT SUPPORTED YET}\\
  ASKI internal, but SPECFEM independent spherical inverison grid
\item (\lcode{TYPE_INVERSION_GRID = scartInversionGrid})\\
  \ASKI internal, but \lcode{SPECFEM} independent cartesian inversion grid:\\
  The values for \ASKI output are stored at all inner GLL points of spectral elements which lie
  inside the block volume defined below by parameters \lcode{ASKI_(cw)(xyz)}.
  \ASKI loactes the coordinates of those points inside the inversion grid cells and computes
  integration weights for them.
\item (\lcode{TYPE_INVERSION_GRID = ecartInversionGrid}) \\
  External inversion grid provided e.g.\ by \lcode{CUBIT}, which may contain tetrahedra, as well as hexahedra.
  As in case of \lcode{ASKI_type_inversion_grid = 2}, \ASKI output is stored at all inner GLL points of elements
  which are inside the volume defined by \lcode{ASKI_(cw)(xyz)}.
  \ASKI locates the wavefield points inside the inversion grid and computes weights.
\item (\lcode{TYPE_INVERSION_GRID = specfem3dInversionGrid}) \\
  Use \lcode{SPECFEM} elements as inversion grid:\\
  Wavefield points are \emph{all} GLL points of an element for elements which are (at least partly) inside the 
  volume defined by \lcode{ASKI_(cw)(xyz)}. Additionally store the jacobians for all wavefield points.
  Assume \lcode{ncell = ntot_wp/(NGLLX*NGLLY*NGLLY)} as the number of inversion grid cells, and the order of 
  wavefield points accordingly (\lcode{do k=1,NGLLZ;} \lcode{do j=1,NGLLY;} \lcode{do i=1,NGLLX;} \lcode{ip=ip+1 ....})
\end{enumerate}
%----------------------------------------------------------------------
\subsubsection*{\lcode{ASKI_(cw)(xyz), ASKI_rot_(XYZ)}}
%----------------------------------------------------------------------
Dependent on \lcode{ASKI_type_inversion_grid}, (a selection of) the following parameters may be used to define a volume 
within which wavefield points are searched for:

First, \lcode{ASKI_wx,ASKI_wy,ASKI_wz} define the total width of a block which is centered in \lcode{x=y=z=0}
E.g.\ the total block extension in x-direction covers all points with\\
\lcode{x >= - 0.5*ASKI_wx} and \lcode{x <=  0.5*ASKI_wx}.\\
Then, \lcode{ASKI_rot_X,ASKI_rot_Y,ASKI_rot_Z} define rotation angles in degrees by which the block is 
rotated (anti-clockwise) about the \lcode{Z}, \lcode{Y} and \lcode{X} coordinate axis, before 
\lcode{ASKI_cx,ASKI_cy,ASKI_cz} define a vector by which the rotated block is shifted (new center of block).

\emph{Be aware}:
\begin{itemize}
\item the parameters for rotation angles \lcode{ASKI_rot_(XYZ)} \emph{must always} be assinged to values! 
  Set to \lcode{0.} if not used.
\item \lcode{scartInversionGrid} only supports \lcode{ASKI_rot_Z} and uses a different definintion of the z-coverage.
\item \lcode{ecartInversionGrid} and \lcode{specfem3dInversionGrid} use \emph{all} rotation angles \lcode{ASKI_rot_(XYZ)}.
\end{itemize}
%
%-------------------------------
% SECTION install kernel output
%#############################################################
\section{Extend \lcodetitle{SPECFEM3D\_Cartesian-2.1} to produce output for \ASKI} \label{extent_to_ASKI}
%#############################################################
%
This section explains how to use the \lcode{SPECFEM3D_Cartesian} software 
\url{http://geodynamics.org/cig/software/specfem3d} as a forward method for \ASKI. In general, a regularly
installed \lcode{SPECFEM3D_Cartesian} version is extended by certain few modifications so it can produce
output for \ASKI. So, \lcode{SPECFEM3D_Cartesian for ASKI} basically has the same requirements and dependencies 
than the \lcode{SPECFEM3D_Cartesian} code, except that it needs a bit more memory and disc space. You should, 
therefore, have sufficient knowledge of how to run the regular \lcode{SPECFEM3D_Cartesian} software. Furthermore, 
you need an installation of \ASKI (obviously).
%
%----------------------------------------------------------------------
\subsection{Download and Dependencies} \label{extent_to_ASKI,sub:download_dependencies}
%----------------------------------------------------------------------
%
The extension package \lcode{SPECFEM3D_Cartesian-2.1 for ASKI} can be downloaded from 
\url{http://www.rub.de/ASKI}.

It assumes a running version of \lcode{SPECFEM3D_Cartesian} on your system which must be capable of a 
certain functionality. If the currently availabe release version of \lcode{SPECFEM3D_Cartesian} does not have 
this functionality, you can download \url{http://www.rub.de/ASKI}. 
This is a basic extract from the \lcode{SPECFEM3D_Cartesian} svn-repository (svn-revision by 9 june 2013, 
\lcode{SPECFEM3D} version \lcode{2.1}) which has been debugged and slightly modified such that it is capable of the 
required functionality. Important functionality is listed in the following, whereby the code references are related 
to the previously mentioned modified svn revision:
\begin{itemize}
\item in case of \lcode{USE_FORCE_POINT_SOURCE = .true.} in \lcode{Par_file} the sources should be interpolated inside the 
  source array and not just set to the closest GLL point (compare \lcode{src/specfem3D/locate_source.f90}, comments \lcode{FS FS})\\
  This, however, is not a strict necessity, you can also use \lcode{SPECFEM3D_Cartesian} codes which do not do this.
%
\item in case of \lcode{USE_FORCE_POINT_SOURCE = .true.} and \\
  \lcode{USE_RICKER_TIME_FUNCTION = .false.} in 
  \lcode{Par_file}, the regular Heaviside source time function (function \lcode{comp_source_time_function}) must
  be used (compare \lcode{src/specfem3D/compute_add_sources_viscoelastic.f90}, \lcode{src/specfem3D/setup_sources_receivers.f90} 
  comments \lcode{FS FS})
%
\item using external models (\lcode{MODEL = external} in \lcode{Par_file}), subroutine \lcode{model_external_values} must 
  be passed the default model values on enter (compare \lcode{src/generate_databases/get_model.f90}, comments \lcode{FS FS})
%
\item the general user interface of the \lcode{SPECFEM3D_Cartesian} code (i.e.\ definition of source mechanisms and receivers, 
  models, output files etc.) must be compatible with \lcode{SPECFEM3D_Cartesian} version \lcode{2.1} by june 2013.
\end{itemize}
%
%----------------------------------------------------------------------
\subsection{Installation}
%----------------------------------------------------------------------
%
Extract the files of the extension package \lcode{SPECFEM3D_Cartesian-2.1 for ASKI} into a subdirectory
of your \ASKI installation path, in the following refered to as \lcode{SPECFEM_for_ASKI}. I.e.\ the directory 
\lcode{SPECFEM_for_ASKI} is contained in directory \ASKI.\\
Install a \lcode{SPECFEM3D_Cartesian} code on your system which meets the requirements as in 
subsection~\ref{extent_to_ASKI,sub:download_dependencies}, in the following the \lcode{SPECFEM3D_Cartesian} 
installation path is refered to as \lcode{SPECFEM3D}.
\begin{itemize}
\item Follow the 9 items in the ``Installation'' section of file \lcode{ASKI/SPECFEM_for_ASKI/README}, whereby the first one
  should already be completed.
\item Adjust variables \lcode{SHELL}, \lcode{BLAS}, \lcode{LAPACK} etc.\ in \lcode{SPECFEM_for_ASKI/Makefile} in the 
  same way as you did in \lcode{ASKI/Makefile} for the installation of \ASKI. The same environment variables are assumed 
  here, too.
\item Run \lcode{make clean} and \lcode{make all} in \lcode{ASKI/SPECFEM_for_ASKI}.
\item Set \lcode{USE_SOURCES_RECVS_Z = .true.} in \lcode{SPECFEM3D/src/shared/constants.h} (or wherever your \lcode{constants.h}
  is).
\item Recompile your \lcode{SPECFEM3D_Cartesian} code by running \lcode{make clean} and \lcode{make all} in \lcode{SPECFEM3D}.
\end{itemize}
%
%-------------------------------
% SECTION GNU Free Documentation License
\newpage
\input{fdl-1.3}
%

\end{document}
